%        File: presenta.tex
%     Created: Mon Sept 16 11:00 AM 2019
% Last Change: Mon Sept 16 11:00 AM 2019
%
%      Author: Carlos Rodríguez

\documentclass[hyperref={pdfpagelabels=false},xcolor=pst,pdf,fragile]{beamer}


\providecommand\thispdfpagelabel[1]{}
\usepackage{lmodern}
\usepackage[utf8]{inputenc}
\usepackage{listings}
\usepackage{hyperref}
\usepackage{subfig}
%\usepackage[spanish]{babel}

\usetheme{Boadilla}
\usefonttheme{serif}

\author{
  Luis, Gutierrez
  \texttt{cinema.nightmare@gmail.com}
  \and
  \\Carlos Rodríguez
  \texttt{carlosrdz.isd@gmail.com}
}
\def\Title{Introducción a python}



\title{\Title}

\date{\today}

\begin{document}
\maketitle

\AtBeginSection[]
{
  \begin{frame}
    \frametitle{Outline}
    \tableofcontents[currentsection]
  \end{frame}
}

\AtBeginSubsection[]
{
  \begin{frame}
    \frametitle{Outline}
    \tableofcontents[currentsection,currentsubsection]
  \end{frame}
}

\section{Comenzando el viaje a al programación}
\begin{frame}
    \frametitle{Constantes y variables}
    \pause
    \begin{itemize}
    \item Una constante es un valor que se mantiene fijo durante todo el programa. 
    Ejemplos: 5, 'a', 38, "www.google.com"
    \item Una variable es un espacio de almacenamiento en memoria que contiene cierta información, la cuál puede ser modificada durante la ejecución del programa.
    Ejemplos: variable, a, a5 = 32, var = "hola"
    \end{itemize}
\end{frame}

\begin{frame}
    \frametitle{Tipos de datos}
    \pause
    \begin{itemize}
    \item Entero (integer)
    \item Flotante (float)
    \item Número complejo
    \item Booleano (boolean)
    \item String
    \item Lista
    \item Tupla
    \item Diccionario
    \end{itemize}
\end{frame}

\begin{frame}
    \frametitle{Operadores aritméticos}
    \pause
    \begin{itemize}
    \item + (suma)
    \item - (resta)
    \item * (multiplicación)
    \item / (división)
    \item \% (módulo)
    \item ** (exponente)
    \item // función suelo
    \end{itemize}
\end{frame}


\begin{frame}
    \frametitle{Operadores lógicos}
    \pause
    \begin{itemize}
    \item \textgreater (mayor)
    \item \textless (menor)
    \item == (igual a)
    \item != (diferente de)
    \item \textgreater= (mayor o igual)
    \item \textless= (menor o igual)
    \end{itemize}
\end{frame}

\begin{frame}
    \frametitle{Operadores lógicos}
    \begin{itemize}
    \item and (y)
    \item or (o)
    \item not
    \item y otros...
    \end{itemize}
\end{frame}

\section{Aterrizando a python}
\begin{frame}
    \frametitle{Hola python!}
    \pause
    \begin{itemize}
    \item \# En este programa desplegamos un saludo
    \item print("Hola grupo de python!")
    \end{itemize}
\end{frame}

\begin{frame}
    \frametitle{Comentarios}
    \pause
    \begin{itemize}
    \item Los comentarios sirven para hacer más entendible el código, tanto para nosotros como para los demás
    \item Python sabe que es un comentario al escribir el hashtag al inicio de la línea
    \item \# Así por ejemplo
    \item Otra forma de escribir un comentario es con triple comilla doble al inicio y al final del comentario
    \item """Este es otro ejemplo de 
    \item comentarios"""
    \end{itemize}
\end{frame}

\section{Tipos de datos (otra vez)}
\begin{frame}
    \frametitle{Especificando a python}
    \pause
    \begin{itemize}
    \item Para que python sepa qué tipo de dato deseamos utilizar, es necesario realizar un "cast" al tipo de dato. En caso de no especificar el tipo de dato que deseamos, python otorgará el que el crea más apropiado a la variable, por ejemplo:
    \item str(texto)
    \item int(numero)
    \item float(texto) \pause \#esto es válido ;)
    \end{itemize}
\end{frame}

\begin{frame}
    \frametitle{Números}
    \pause
    \begin{itemize}
    \item Los dos tipos principales de números en python son los enteros (integers) y reales (floats)
    \item En caso de no especificar el tipo de número (o de dato) python decide cuál es el más apropiado (a veces se equivoca)
    \item Integers (int): 2, 5, 4000, -2
    \item Floats (float): 2.5, 2.9, -3.1
    \end{itemize}
\end{frame}

\begin{frame}
    \frametitle{Texto}
    \pause
    \begin{itemize}
    \item Una variable que almacena texto se llama cadena o string, y para que python las reconozca deben estar en comillas simples o dobles... o triples si necesita varias líneas
    Ejemplo: "Texto de ejemplo"
    \item el operador + une dos strings...
    \end{itemize}
\end{frame}

\section{Interactuando con el usuario}
\begin{frame}
    \frametitle{input y print}
    \pause
    \begin{itemize}
    \item Para desplegar algún dato de nuestro programa, basta con utilizar el comando print, mientras que para asignar algún valor dado por el usuario a una variable, se utiliza el comando input. Ejemplos:
    \item print("Desplegando un texto")
    \item var=input("Ingresa un texto")
    \item print(var)
    \end{itemize}
\end{frame}

\section{Tipos de datos (última continuación)}
\begin{frame}
    \frametitle{Arreglos (colecciones)}
    \pause
    \begin{itemize}
    \item Existen 4 tipos de arreglos en python (legacy), los cuáles permiten almacenar varios datos en una sola variable:
    \item Listas (List): Ordenada e intercambiable. Permite miembros duplicados.
    \item Tuplas (Tuple): Coleccion ordenada y no intercambiable. Permite miembros duplicados.
    \item Conjuntos (Set): Desordenada y no indexada. No permite miembros duplicados.
    \item Diccionarios (Dictionary): Desordenada, intercambiable e indexada. No permite miembros duplicados.
    \end{itemize}
\end{frame}

\begin{frame}
    \frametitle{Listas}
    \pause
    \begin{itemize}
    \item La lista es una colección ordenada e intercambiable.
    \item Para crear una lista en python se escriben sus elementos entre corchetes [a, b, c,"hola"].
    \item mi\textunderscore lista=["un\textunderscore elemento","otro\textunderscore elemento","tercer\textunderscore elemento", "otro", "otro", "ya", "son", "muchos"]
    \item print(mi\textunderscore lista)
    \item Para accesar a una lista, especificamos el número de indexación al que queremos accesar (el primer elemento se indentifica con el número 0).
    \item print(mi\textunderscore lista[1])
    \end{itemize}
\end{frame}

\begin{frame}
    \frametitle{Listas}
    \pause
    \begin{itemize}
    \item python permite indexamiento negativo... \pause no lo usen )=
    \item Es posible accesar a una sublista de un rango especificado
    \item print(mi\textunderscore lista[2:5])
    \item print(:5)
    \end{itemize}
\end{frame}

\begin{frame}
    \frametitle{Operaciones de listas}
    \pause
    \begin{itemize}
    \item Las operaciones más habituales de Python son las siguientes:
    \item lista[i]: Devuelve el elemento que está en la posición i de la lista.
    \item lista.pop(i): Devuelve el elemento en la posición i de una lista y luego lo borra.
    \item lista.append(elemento): Añade elemento al final de la lista.
    \item lista.insert(i, elemento): Inserta elemento en la posición i.
    \item lista.extend(lista2): Fusiona lista con lista2.
    \item lista.remove(elemento): Elimina la primera vez que aparece elemento.
    \end{itemize}
\end{frame}


\begin{frame}
    \frametitle{Diccionarios}
    \pause
    \begin{itemize}
    \item Un diccionario (tal como los convencionales) es una palabra que tiene asociado "algo", y a diferencia de las listas, los diccionarios no tienen orden.
    \item Para crear un diccionario se ponen sus elementos entre llaves \{"a":"Almidon","b":"bueno",:\}. se les denomina llaves (keys) a las palabras (a y b) y valores a sus definiciones (Almidon y bueno). No es posible tener dos llaves iguales, aunque si dos valores.
    \item Ejemplo: 
    \item diccionario = \{'Piloto 1':'Fernando Alonso', 'Piloto 2':'Kimi Raikkonen', 'Piloto 3':'Felipe Massa'\}
    \item print(diccionario)
    \end{itemize}
\end{frame}

\begin{frame}
    \frametitle{Operaciones más comunes de los diccionarios}
    \pause
    \begin{itemize}
    \item diccionario.get(‘key’): Devuelve el valor que corresponde con la key introducida.
    \item diccionario.pop(‘key’): Devuelve el valor que corresponde con la key introducida, y luego borra la key y el valor.
    \item diccionario.update(\{‘key’:’valor’\}): Inserta una determinada key o actualiza su valor si ya existiera.
    \item "key" in diccionario: Devuelve verdadero (True) o falso (False) si la key (no los valores) existe en el diccionario.
    \item "definicion" in diccionario.values(): Devuelve verdadero (True) o falso (False) si definición existe en el diccionario (no como key).
    \end{itemize}
\end{frame}

\section{Conjuntos}
\begin{frame}
    \frametitle{Conjuntos}
    \pause
    \begin{itemize}
    \item También existen los conjuntos (sets) aunque son menos utilizados.
    \item Tal como los diccionarios, se crean usando llaves, pero sus elementos se separan por coma como si se tratara de una lista
    \item Ejemplo: conjunto = \{'Fernando Alonso', 'Kimi Raikkonen', 'Felipe Massa'\}
    \item Los sets permiten realizar operacionas matemáticas típicas de conjuntos como unión, intersección, etc.
    \end{itemize}
\end{frame}

\begin{frame}
    \frametitle{Operaciones más comunes en conjuntos}
    \pause
    \begin{itemize}
    \item  A | B: Unión entre el conjunto A y B (Los elementos del conjunta A y los elementos del conjunto B)
    \item A \& B: Intersección entre el conjunto A y B (los elementos que están en ambos conjuntos)
    \item A – B: Diferencia entre el conjunto A y B (los elementos que están en A pero no están en B)
    \item A\textasciicircum B: Diferencia simétrica entre el conjunto A y B (los elementos que están en A o en B pero no en los dos)
    \end{itemize}
\end{frame}

\section{Indentación (sangría) en python}
\begin{frame}
    \frametitle{Indentación}
    \pause
    \begin{itemize}
    \item Python considera es sensible a la indentación, por lo que es necesario que  sus líneas de código se encuentren agrupadas con el mismo número de espacios a la izquierda de cada línea. Es recomendado utilizar bloques de cuatro espacios, sin embargo cualquier otra cantidad de espacios (o tabuladores) pueden ser utilizados de igual forma (no se recomienda).
    \end{itemize}
\end{frame}

\section{Control de flujo y condicionales}
\begin{frame}
    \frametitle{Control de flujo y condicionales}
    \pause
    \begin{itemize}
    \item Todo programa con una ligera complejidad necesita "tomar decisiones" en algunas bifurcaciones del problema, donde, dependiendo de alguna condición se realiza una u otra acción.
    \item Esta bifurcación se lleva a cabo con el comando if (condición principal), con sus opcionales elif (condiciones adicionales, pueden ponerse tantos como sean deseados) y else (el default en caso de no cumplirse ninguno).
    \end{itemize}
\end{frame}

\begin{frame}
    \frametitle{Ejemplo}
    \pause
    \begin{itemize}
    \item Revisar ejercicio "ifs.py"
    \end{itemize}
\end{frame}

\begin{frame}
    \frametitle{Condiciones en python}
    \pause
    \begin{itemize}
    \item Las condiciones utilizadas con más frecuencia son:
    \item a == b –\textgreater Indica si a es igual a b
    \item a \textless b
    \item a \textgreater b
    \item not –\textgreater NO: niega la condición que le sigue.
    \item and –\textgreater Y: junta dos condiciones que tienen que cumplirse las dos
    \item or –\textgreater O: junta dos condiciones y tienen que cumplirse alguna de las dos.
    \end{itemize}
\end{frame}

\begin{frame}
    \frametitle{while en python}
    \pause
    \begin{itemize}
    \item Cuando una tarea debe repetirse hasta que cierta condición se cumpla (no sabemos cuántas veces se repetirá), es recomendado utilizar el comando while, que se usa de la siguiente forma:
    \item vuelta=1
    \item while vuelta \textless 10:
    \item \quad \quad \quad print("vuelta "+str(vuelta))
    \item \quad \quad \quad vuelta=vuelta+1
    \end{itemize}
\end{frame}

\begin{frame}
    \frametitle{for en python}
    \pause
    \begin{itemize}
    \item En ocasiones necesitamos repetir varias veces una determinada tarea, pero conocemos la cantidad de veces que la repetiremos. Para esto es recomendado utilizar en Python el comando for, que se usa de la siguiente forma:
    \item for vuelta in range(1,10):
    \item \quad \quad \quad print("vuelta "+str(vuelta))
    \item PD. En el caso del for (a diferencia del while) no es posible realizar un loop infinito.
    \item es posible utilizar el for con cualquier objeto que pueda ser iterado (recorrer sus elementos)
    \end{itemize}
\end{frame}

\section{Módulos y Funciones}

\begin{frame}
    \frametitle{No reinventes la rueda}
    \pause
    \begin{itemize}
    \item Un módulo es un programa que viene en un "paquete" de python y existen para contener rutinas que son utilizadas habitualmente y que no sea necesario reescribirlas cada que son necesitadas. Estos módulos son creados muchas veces por programadores como tu, y se comparten en Github principalmente asi que si hay alguna rutina que uses mucho y no existe ningun paquete que la contenga, es tu oportunidad de brillar ;)
    \item Gracias a estos módulos es posible realizar cosas complejas en pocas líneas de código
    \end{itemize}
\end{frame}

\begin{frame}
    \frametitle{Instalando un módulo}
    \pause
    \begin{itemize}
    \item Los módulos son instalados de forma distinta en los diversos sistemas operativos, en el caso de Linux, basta con bajar el programa pip y escribir en nuestra línea de comandos "pip install \textless módulo\textgreater " para instalarlo... en algunas distribuciones es incluso más fácil que eso, pero recuerda que internet es tu amigo (internet, google no)
    \end{itemize}
\end{frame}

\begin{frame}
    \frametitle{Módulos "legacy" más usados}
    \pause
    \begin{itemize}
    \item os
    \item datetime
    \item time
    \item sys
    \item locale
    \item MySQLdb
    \item Cada uno de estos módulos contiene funciones ya incluídas que pueden ser llamadas si esl módulo se encuentra instalado y cargado.
    \end{itemize}
\end{frame}

\begin{frame}
    \frametitle{Invocando funciones de módulos}
    \pause
    \begin{itemize}
    \item Para llamar a una función de un módulo es necesario primero cargar el módulo a nuestro programa, con el comando:
    \item import modulo
    \item Posteriormente basta con llamar a la función deseada en la forma:
    \item modulo.funcion()
    \end{itemize}
\end{frame}

\begin{frame}
    \frametitle{Declaración de funciones}
    \pause
    \begin{itemize}
    \item Cuando existe alguna rutina que queremos repetir constantemente durante nuestro programa, es útil crear una función, la cuál nos permitirá llamar esta rutina con un solo comando (en lugar de escribir todo). Ejemplo:
    \item def myfunc(): \#Aquí especificamos el nombre de la función
    \item \quad \quad \quad x = 300
    \item \quad \quad \quad print(x)
    \item myfunc() \#Esto llama a la función
    \end{itemize}
\end{frame}

\begin{frame}
    \frametitle{Scope}
    \pause
    \begin{itemize}
    \item Una variable se encuentra disponible solamente dentro de la región donde fue creada, y a esto de le denomina "scope" o "alcance".
    \item Una variable que es creada dentro de una función pertenece solamente al "scope local" de esa función, y solamente podrá ser usada dentro de esa función
    \item ***Recuerda que la indentación afecta al scope de las variables
    \end{itemize}
\end{frame}

\section{Archivos I/O}
\begin{frame}
    \frametitle{Interactuando con archivos}
    \pause
    \begin{itemize}
    \item Con Python podemos manipular archivos (leer, escribir...) de una forma bastante sencilla
    \end{itemize}
\end{frame}

\begin{frame}
    \frametitle{Creando archivos de texto}
    \pause
    \begin{itemize}
    \item Para crear un archivo hay que seguir los siguientes pasos:
    \item Paso 1) Abre un archivo y asignalo a una variable (con la opción "w+")
    \item Paso 2) Escribimos la informacion que queramos que contenga el archivo
    \item Paso 3) Cerramos el archivo
    \end{itemize}
\end{frame}

\begin{frame}
    \frametitle{Creando archivos de texto (Ejemplo)}
    \pause
    \begin{itemize}
    \item Paso 1) f=open("archivo.txt", "w+")
    \item Paso 2) for i in range(10):
    \item \quad \quad \quad f.write("This is line \%d \textbackslash r \textbackslash n" \% (i+1))
    \item Paso 3) f.close()
    \end{itemize}
\end{frame}

\begin{frame}
    \frametitle{Escribir datos sobre un archivo}
    \pause
    \begin{itemize}
    \item Para escribir sobre un archivo hay que seguir los siguientes pasos:
    \item Paso 1) Abre un archivo y asignalo a una variable (con la opción "a+")
    \item Paso 2) Escribe lo que desees con el comando \textless variable \textgreater.write("texto")
    \item Paso 3) Cerramos el archivo
    \end{itemize}
\end{frame}

\begin{frame}
    \frametitle{Escribir datos sobre un archivo (Ejemplo)}
    \pause
    \begin{itemize}
    \item Paso 1) f=open("archivo.txt", "a+")
    \item Paso 2) for i in range(2):
    \item \quad \quad \quad f.write("Appended line \%d \textbackslash r \textbackslash n" \% (i+1))
    \item Paso 3) f.close()
    \end{itemize}
\end{frame}

\begin{frame}
    \frametitle{Leer un archivo}
    \pause
    \begin{itemize}
    \item Para Leer un archivo hay que seguir los siguientes pasos:
    \item Paso 1) Abre un archivo y asignalo a una variable (con la opción "r+")
    \item Paso 2) Leemos la información deseada con el comando \textless variable \textgreater .read()... y es probable que quieras almacenarla en alguna variable
    \item Paso 3) Cerramos el archivo
    \end{itemize}
\end{frame}

\begin{frame}
    \frametitle{Leer un archivo (Ejemplo)}
    \pause
    \begin{itemize}
    \item Paso 1) f=open("archivo.txt", "r")
    \item Paso x) if f.mode == 'r': \#esto es opcional, pero recomendado
    \item Paso 2) contents =f.read()
    \item Paso 3) f.close()
    \end{itemize}
\end{frame}

\section{Test de cuadrito}
\begin{frame}[fragile]
  \frametitle{frame con cuadrito }
  \begin{itemize}
	\item (=
	\pause
  \end{itemize}

    \begin{block}{Bibliografías} % alertblock, exampleblock
	  \begin{lstlisting}
https://www.tutorialspoint.com/python/python_basic_operators.htm
https://www.tutorialsteacher.com/python/python-comparison-and-logical-operators
https://www.learnpython.org/es/Hello,%20World!
https://www.w3schools.com/python/python_lists.asp
https://www.guru99.com/reading-and-writing-files-in-python.html
https://www.tutorialpython.com/modulos-python/
	  \end{lstlisting}
  \end{block}
  
\end{frame}

\end{document}


