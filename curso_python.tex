%        File: presenta.tex
%     Created: Mon Sept 16 11:00 AM 2019
% Last Change: Mon Sept 16 11:00 AM 2019
%
%      Author: Carlos Rodríguez

\documentclass[hyperref={pdfpagelabels=false},xcolor=pst,pdf,fragile]{beamer}


\providecommand\thispdfpagelabel[1]{}
\usepackage{lmodern}
\usepackage[utf8]{inputenc}
\usepackage{listings}
\usepackage{hyperref}
\usepackage{subfig}
%\usepackage[spanish]{babel}

\usetheme{Boadilla}
\usefonttheme{serif}

\author{
  Luis, FirstName1
  \texttt{luis.last1@gmail.com}
  \and
  \\Carlos Rodríguez
  \texttt{carlosrdz.isd@gmail.com}
}
\def\Title{Introducción a python}



\title{\Title}

\date{\today}

\begin{document}
\maketitle

\AtBeginSection[]
{
  \begin{frame}
    \frametitle{Outline}
    \tableofcontents[currentsection]
  \end{frame}
}

\AtBeginSubsection[]
{
  \begin{frame}
    \frametitle{Outline}
    \tableofcontents[currentsection,currentsubsection]
  \end{frame}
}

\section{Comenzando el viaje a al programación}
\begin{frame}
    \frametitle{Constantes y variables}
    \pause
    \begin{itemize}
    \item Una constante es un valor que se mantiene fijo durante todo el programa. 
    Ejemplos: 5, 'a', 38, "www.google.com"
    \item Una variable es un espacio de almacenamiento en memoria que contiene cierta información, la cuál puede ser modificada durante la ejecución del programa.
    Ejemplos: variable, a, a5
    \end{itemize}
\end{frame}

\begin{frame}
    \frametitle{Tipos de datos}
    \pause
    \begin{itemize}
    \item Entero (integer)
    \item Flotante (float)
    \item Número complejo
    \item Booleano (boolean)
    \item String
    \item Lista
    \item Tupla
    \item Diccionario
    \end{itemize}
\end{frame}

\begin{frame}
    \frametitle{Operadores aritméticos}
    \pause
    \begin{itemize}
    \item + (suma)
    \item - (resta)
    \item * (multiplicación)
    \item / (división)
    \item (módulo)
    \item ** (exponente)
    \item función suelo
    \end{itemize}
\end{frame}


\begin{frame}
    \frametitle{Operadores lógicos}
    \pause
    \begin{itemize}
    \item > (mayor)
    \item  (menor)
    \item == (igual a)
    \item != (diferente de)
    \item >= (mayor o igual)
    \item  (menor o igual)
    \item and (y)
    \item or (o)
    \item not (no)
    \end{itemize}
\end{frame}

\begin{frame}
    \frametitle{Operadores lógicos}
    \begin{itemize}
    \item  y
    \item  o
    \item ~ not
    \item y otros...
    \end{itemize}
\end{frame}

\section{Aterrizando a python}
\begin{frame}
    \frametitle{Hola python!}
    \pause
    \begin{itemize}
    \item Decimos hola
    \end{itemize}
\end{frame}


\begin{frame}
    \frametitle{Comentarios}
    \pause
    \begin{itemize}
    \item Escribir el hashtag al inicio de la línea
    \item # Así por ejemplo
    \item Escribir triple comilla doble al inicio y al final del comentario
    \item """Este es otro ejemplo de comentarios"""
    \end{itemize}
\end{frame}









\section{Tipos de datos (otra vez)}
\begin{frame}
    \frametitle{Números}
    \pause
    \begin{itemize}
    \item Los dos tipos principales de números en python son los enteros (integers) y reales (floats)
    \item En cado de no especificar el tipo de número (o de dato) python decide cuál es el más apropiado (a veces se equivoca)
    \item Integers (int): 2, 5, 4000, -2
    \item Floats (float): 2.5, 2.9, -3.1
    \end{itemize}
\end{frame}

\begin{frame}
    \frametitle{Texto}
    \pause
    \begin{itemize}
    \item Una variable que almacena texto se llama cadena o string, y para que python las reconozca deben estar en comillas simples o dobles... o triples si necesita varias líneas
    Ejemplo: "Texto de ejemplo"
    \item el operador + une dos strings...
    \end{itemize}
\end{frame}

\section{Interactuando con el usuario}
\begin{frame}
    \frametitle{Print y get}
    \pause
    \begin{itemize}
    \item Decimos hola
    \end{itemize}
\end{frame}

\begin{frame}
    \frametitle{Texto}
    \pause
    \begin{itemize}
    \item Una variable que almacena texto se llama cadena o string, y para que python las reconozca deben estar en comillas simples o dobles... o triples si necesita varias líneas
    Ejemplo: "Texto de ejemplo"
    \end{itemize}
\end{frame}

\section{Arreglos (colecciones)}
\begin{frame}
    \frametitle{Arreglos}
    \pause
    \begin{itemize}
    \item Listas
    \item Diccionarios
    \item Conjuntos (sets)
    \item Tuplas
    \end{itemize}
\end{frame}

\begin{frame}
    \frametitle{Listas}
    \pause
    \begin{itemize}
    \item Listas
    \item Diccionarios
    \item Conjuntos (sets)
    \item Tuplas
    \end{itemize}
\end{frame}

\begin{frame}
    \frametitle{Diccionarios}
    \pause
    \begin{itemize}
    \item Listas
    \item Diccionarios
    \item Conjuntos (sets)
    \item Tuplas
    \end{itemize}
\end{frame}

\section{Control de flujo y condicionales}
\begin{frame}
    \frametitle{Control de flujo y condicionales}
    \pause
    \begin{itemize}
    \item Decimos hola
    \end{itemize}
\end{frame}

\section{Funciones}
\begin{frame}
    \frametitle{Scope}
    \pause
    \begin{itemize}
    \item Decimos hola
    \end{itemize}
\end{frame}

\begin{frame}
    \frametitle{Llamando una función}
    \pause
    \begin{itemize}
    \item Decimos hola
    \end{itemize}
\end{frame}

\begin{frame}
    \frametitle{Declaración de funciones}
    \pause
    \begin{itemize}
    \item Decimos hola
    \end{itemize}
\end{frame}

\section{I/O}
\begin{frame}
    \frametitle{Interactuando con el archivo}
    \pause
    \begin{itemize}
    \item Decimos hola
    \end{itemize}
\end{frame}

\begin{frame}
    \frametitle{Interactuando con la propia computadora}
    \pause
    \begin{itemize}
    \item Decimos hola
    \end{itemize}
\end{frame}

\section{Test de cuadrito}
\begin{frame}[fragile]
  \frametitle{frame con cuadrito }
  \begin{itemize}
	\item esta presentacion tiene un cuadrito
	\pause
  \end{itemize}

    \begin{block}{Open Definition} % alertblock, exampleblock
	  \begin{lstlisting}
https://www.tutorialspoint.com/python/python_basic_operators.htm
https://www.tutorialsteacher.com/python/python-comparison-and-logical-operators
https://www.learnpython.org/es/Hello,%20World!
https://www.w3schools.com/python/python_lists.asp
	  \end{lstlisting}
  \end{block}
  
\end{frame}


\end{document}


